% This file is part of GUFI, which is part of MarFS, which is released
% under the BSD license.
%
%
% Copyright (c) 2017, Los Alamos National Security (LANS), LLC
% All rights reserved.
%
% Redistribution and use in source and binary forms, with or without modification,
% are permitted provided that the following conditions are met:
%
% 1. Redistributions of source code must retain the above copyright notice, this
% list of conditions and the following disclaimer.
%
% 2. Redistributions in binary form must reproduce the above copyright notice,
% this list of conditions and the following disclaimer in the documentation and/or
% other materials provided with the distribution.
%
% 3. Neither the name of the copyright holder nor the names of its contributors
% may be used to endorse or promote products derived from this software without
% specific prior written permission.
%
% THIS SOFTWARE IS PROVIDED BY THE COPYRIGHT HOLDERS AND CONTRIBUTORS "AS IS" AND
% ANY EXPRESS OR IMPLIED WARRANTIES, INCLUDING, BUT NOT LIMITED TO, THE IMPLIED
% WARRANTIES OF MERCHANTABILITY AND FITNESS FOR A PARTICULAR PURPOSE ARE DISCLAIMED.
% IN NO EVENT SHALL THE COPYRIGHT HOLDER OR CONTRIBUTORS BE LIABLE FOR ANY DIRECT,
% INDIRECT, INCIDENTAL, SPECIAL, EXEMPLARY, OR CONSEQUENTIAL DAMAGES (INCLUDING,
% BUT NOT LIMITED TO, PROCUREMENT OF SUBSTITUTE GOODS OR SERVICES; LOSS OF USE,
% DATA, OR PROFITS; OR BUSINESS INTERRUPTION) HOWEVER CAUSED AND ON ANY THEORY OF
% LIABILITY, WHETHER IN CONTRACT, STRICT LIABILITY, OR TORT (INCLUDING NEGLIGENCE
% OR OTHERWISE) ARISING IN ANY WAY OUT OF THE USE OF THIS SOFTWARE, EVEN IF
% ADVISED OF THE POSSIBILITY OF SUCH DAMAGE.
%
%
% From Los Alamos National Security, LLC:
% LA-CC-15-039
%
% Copyright (c) 2017, Los Alamos National Security, LLC All rights reserved.
% Copyright 2017. Los Alamos National Security, LLC. This software was produced
% under U.S. Government contract DE-AC52-06NA25396 for Los Alamos National
% Laboratory (LANL), which is operated by Los Alamos National Security, LLC for
% the U.S. Department of Energy. The U.S. Government has rights to use,
% reproduce, and distribute this software.  NEITHER THE GOVERNMENT NOR LOS
% ALAMOS NATIONAL SECURITY, LLC MAKES ANY WARRANTY, EXPRESS OR IMPLIED, OR
% ASSUMES ANY LIABILITY FOR THE USE OF THIS SOFTWARE.  If software is
% modified to produce derivative works, such modified software should be
% clearly marked, so as not to confuse it with the version available from
% LANL.
%
% THIS SOFTWARE IS PROVIDED BY LOS ALAMOS NATIONAL SECURITY, LLC AND CONTRIBUTORS
% "AS IS" AND ANY EXPRESS OR IMPLIED WARRANTIES, INCLUDING, BUT NOT LIMITED TO,
% THE IMPLIED WARRANTIES OF MERCHANTABILITY AND FITNESS FOR A PARTICULAR PURPOSE
% ARE DISCLAIMED. IN NO EVENT SHALL LOS ALAMOS NATIONAL SECURITY, LLC OR
% CONTRIBUTORS BE LIABLE FOR ANY DIRECT, INDIRECT, INCIDENTAL, SPECIAL,
% EXEMPLARY, OR CONSEQUENTIAL DAMAGES (INCLUDING, BUT NOT LIMITED TO, PROCUREMENT
% OF SUBSTITUTE GOODS OR SERVICES; LOSS OF USE, DATA, OR PROFITS; OR BUSINESS
% INTERRUPTION) HOWEVER CAUSED AND ON ANY THEORY OF LIABILITY, WHETHER IN
% CONTRACT, STRICT LIABILITY, OR TORT (INCLUDING NEGLIGENCE OR OTHERWISE) ARISING
% IN ANY WAY OUT OF THE USE OF THIS SOFTWARE, EVEN IF ADVISED OF THE POSSIBILITY
% OF SUCH DAMAGE.



\section{Deployment}
\label{sec:deploy}
GUFI functionality as described so far has been local. However, users
are not expected to log into the machine with the indexes with an
interactive shell. Instead, they are expected to call wrapper scripts
to forward commands to the server (installed with the client RPM).

\subsection{RPMs}
When client building is enabled and the rpms built (\texttt{make
  package}), two rpms will be created. The server rpm will contain
GUFI as described so far. The client rpm will contain executables that
should be installed on the client node. The user executables on the
client side simply forward commands to the server. A jailed shell
should be set up on the server to prevent arbitrary command execution
- see Section~\ref{sec:gufi_jail} for details.

\subsection{SSH}
Normal ssh, using passwords or user keys (\texttt{\textasciitilde
  /.ssh/id\_rsa}), should be disabled on the server. Instead, ssh
should be done using host based authentication
(\texttt{/etc/ssh/ssh\_host\_rsa\_key}).

\subsection{Runtime Configuration}
\gufiquery provides the core GUFI functionality. However it is not
expected to be used by users, as the syntax of \gufiquery is somewhat
unwieldy, and users are not expected to understand how GUFI
works. Wrappers were created in order to provide users (and
administrators) with predetermined queries that are used often and
extract useful information (see the user guide for more details). The
server has the actual implementation of the wrappers while the clients
have command forwarders.

These wrappers require a valid GUFI configuration file to function
correctly. The path to the configuration file can be set at configure
time with the \texttt{CONFIG\_FILE} CMake variable (defaults to
\guficonfigfile). The configuration file will be readable by all users
but should only be writable by the administrators (664 permissions) in
order to prevent non-admin users from changing how GUFI runs.

Configuration files are simple text files containing lines of
\texttt{<key>=<value>}. Duplicate configuration keys are overwritten
by the bottom-most line. Empty lines and lines starting with \# are
ignored. The server and client require different configuration
values. Example configuration files can be found in the
\texttt{config} directory of the GUFI source, as well as in the build
directory. The corresponding examples will also be installed with
GUFI.

If a server and client exist on the same node, the server and client
configuration file may be combined into one file.

\subsubsection{Server}
\begin{tabular}{| l | l | l |}
  \hline
  Key & Value Type & Purpose \\
  \hline
  Threads & Positive Integer & Number of threads to use \\
  & & when running \gufiquery. \\
  \hline
  Query & File Path & The absolute path of \gufiquery. \\
  \hline
  Stat & File Path & The absolute path of \gufistatbin. \\
  \hline
  IndexRoot & Directory Path & The absolute path of a GUFI tree \\
  & & (or parent of multiple indexes). \\
  \hline
  OutputBuffer & Non-negative Integer & The size of each per-thread
  buffer \\
  & & used to buffer writes to stdout. \\
  & & Setting this too low will affect \\
  & & performance. Setting this too high \\
  & & will use too much resources. \\
  \hline
\end{tabular}

\subsubsection{Client}
\begin{tabular}{| l | l | l |}
  \hline
  Key Value & Type & Purpose \\
  \hline
  Server & URI & The URI of the server (IP address, URL, etc.) \\
  \hline
  Port & Non-negative integer & Port number \\
  \hline
\end{tabular}

\subsection{\gufijail}
\label{sec:gufi_jail}
\gufijail is a simple script that limits the commands users logging in
with ssh are able to run to only a subset of GUFI
commands. \texttt{/etc/ssh/sshd\_config} should be modified with the
following lines:

\begin{verbatim}
    Match Group gufi
        ForceCommand /path/to/gufi_jail
\end{verbatim}
